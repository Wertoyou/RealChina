\documentclass[a4paper,10pt]{article}
\usepackage[top=1in,bottom=1in,left=1in,right=1in]{geometry}
\usepackage{fullpage}
\usepackage[english,swedish]{babel}
\usepackage[T1]{fontenc} 
\usepackage[utf8]{inputenc}
\usepackage{amsmath}
\usepackage{amssymb}
\usepackage{amsthm}
\usepackage{amsfonts}
\usepackage[pdftex]{graphicx}
\usepackage{fancyvrb}
\usepackage{array}
\usepackage{fancyhdr}
\usepackage{courier}
\usepackage{booktabs}
\usepackage{paralist}
\usepackage{xfrac}
\usepackage{siunitx} % Provides the \SI{}{} and \si{} command for typesetting SI units
\usepackage{graphicx} % Required for the inclusion of images
\usepackage{amsmath} % Required for some math elements 
\usepackage{tikz}
% \usetikzlibrary{arrows.meta}
% \usepackage{cite}
\usepackage{enumitem}
\usepackage{float}
\usepackage{wrapfig}
\usepackage{multicol}
\usepackage{caption}
\usepackage{svg}
% \usepackage{subcaption}
% \usepackage{natbib}
\usepackage[citestyle=verbose-ibid,bibstyle=numeric,backend=bibtex,sorting=nty]{biblatex}
\addbibresource{reference.bib}
\usepackage{csquotes}
\usepackage{pdfpages}
\usepackage{appendix}
\usepackage{url}
\usepackage{hyperref}
\usetikzlibrary{scopes}
\usepackage[parfill]{parskip}
\setlength\parindent{0pt}
%\urlstyle{same}
%\setlength{\parskip}{10pt}

\bibliography{sample}

% Ger en titelsida för bilagor
\renewcommand*{\appendixpagename}{\newpage Bilagor} 
\renewcommand*{\appendixtocname}{Bilagor} 




% Gör titel
% \usepackage{titlesec}
% \titleformat{\section}{\normalfont\Large\bfseries}{\S \thesection}{1em}{}
\title{\Huge\bf{Taiwan}\\}
\author{\emph{Björn Thorén, Johannes Agestam}}
\date{\today}
\pagenumbering{gobble}


\begin{document}
%\pagenumbering{gobble}
%\null  % Empty line
%\nointerlineskip  % No skip for prev line
%\vfill
%\let\snewpage \newpage
%\let\newpage \relax
\maketitle
%\let \newpage \snewpage
%\vfill 
% \break % page break

%\newpage

% Det är viktigt att Ni kan urskilja viktiga orsaker/faktorer till konfliktens uppkomst, dvs dess historia i tid och rum.
% Ni skall också redogöra vad som håller konflikten vid liv idag, vilka aktörer som är inblandade.
% Hur tror Ni att det blir i framtiden/ har freden/konfliktlösning någon chans?
% Hur har just ert land/konfliktområdet påverkats av globaliseringen?
% Hur skulle alternativen se ut?


%\section*{Introduktion}
Taiwan är en ö i ostasien som ligger cirka 180 kilometer ost om Kina. Öns yta är cirka 36 000 kvadratkilometer, vilket är lite större än Belgiens yta och lite mindre än Nederländernas. Ön har cirka 23 miljoner invånare, varav över 9 miljoner bor i eller runt Taipei, öns största stad. 

Ända sedan det kinesiska inbördeskriget 1946 - 1949 har Kina haft två regeringar som båda anser sig vara landets enda legitima regim. Den ena, kommunistdiktaturen Folkrepubliken Kina, styr hela Kina förutom Taiwan, som styrs av den demokratiska Republiken Kina.



\section*{Historia}
Taiwans urinvånare anlände till ön från det kinesiska fastlandet för över 5000 år sedan och utgör idag cirka två procent av befolkningen. Nederländska och Spanska handelsmän anlände till ön och byggde bosättningar 1623 respektive 1626. Nederländska trupper underkuvade stora delar av urbefolkningen och tillsammans kunde de driva ut spanjorerna år 1642. Därefter kunde de Nederländska kolonisatörerna styra ön obehindrat. De försökte utbilda och kristna urbefolkningen med begränsad framgång. De försökte även beskatta invånarna samt odla sockerrör och ris. Den bördiga jorden gjorde odlingen lönsam men den krävde mycket arbetskraft vilket ledde till massinvandring av hankineser från det kinesiska fastlandet. 1662 erövrades den Nederländska bosättningen av en militär expedition från fastlandet, som styrdes av Mingdynastin. Mingdynastin kollapsade efter en invasion från Manchuriet och ersattes av den manchuriska Qingdynastin. Krig och svält på fastlandet samt löften om gratis land i utbyte mot militärtjänst ledde till ytterligare invandring till Taiwan, som fortfarande styrdes av Minglojalister. 1683 kapitulerade ön till Qingdynastin och integrerades i kejsardömet. 1895 erövrades ön av Japan, som såg det som en utmärkt bas för vidare expeditioner mot fastlandet och sydostasien.

Under den andra halvan av 1800-talet hade Qing drabbats av ett stort antal utländska expeditioner. Europeiska kolonisatörer hade erövrat Hong Kong och Macau. Opiumkrigen hade tvingat kinesiska hamnar att öppna sig för utländsk handel med omfattande opiummissbruk omkring de större hamnstäderna som följd. De större västerländska länderna hade fått egna självstyrda områden i de viktigaste handelsstäderna och västerländska medborgare hade väldigt priviligerad status i Kina. Ytterligare krig mot Japan och Ryssland hade lett till förlusten av Taiwan, Manchuriet och Korea. Militära under så pass lång tid hade gjort den Manchuriska regeringen väldigt impopulär, men framförallt var folket missnöjt med hur landet styrdes. 

Missnöjet med det växande västerländska inflytandet samt västerlänningars priviligerade status samt deras insmuggling av opium var mycket utbrett. Dessutom var många missnöjda med att den egna regeringen var utlänningar, då de stammade från Manchuriet. Regeringen var fylld av manchurier och hankineser behandlades som andra klassens medborgare som inte kunde inneha höga poster. Korruption var mycket utbrett och den lokala eliten hade väldigt mycket makt, centralregeringen hade svårt att styra landet. 

1898 - 1901 pågick Boxerupproret, där kristna och västerlänningar samt alla som misstänktes ha samröre med dem mördades. De västerländska makterna skickade soldater för att slå ner upproret med våld. När kejsarinnan gjorde motstånd intogs huvudstaden och hon tvingades fly. Hon fick senare återvända men tvingades till ännu fler eftergifter, inklusive stora skadestånd. Hon inledde omfattande reformer för att modernisera samhället men dog kort därefter och efterträddes av en svag kejsare som helt styrdes av sina rådgivare. Den manchuriska eliten vägrade alla reformer som skulle ta ifrån dem makt och avbröt moderniseringen.

1911 inleddes den kinesiska revolutionen genom en revolt i södra Kina. Allteftersom regeringen misslyckades med att slå ner upproret blev man mer positivt inställd till en begränsad monarki och demokratiska reformer, men det var för sent. Allt fler provinser anslöt sig till upproret och 1912 abdikerade kejsaren efter att huvudstaden intagits. Kejsarens abdikering ledde till skapandet av Republiken Kina i Nanjing, men den nya regeringen kunde inte styra landet. I det maktvakuum som uppstod tog lokala krigsherrar makten, och de vägrade lyda centralregeringen. Regeringen fokuserade på att ena landet och ta tillbaka kontrollen från krigsherrarna, de utlovade reformerna uteblev.

1921 grundades det kinesiska kommunistpartiet, som ursprungligen var en del av det styrande nationalistpartiet KMT, som grundades av revolutionsledarna vid republikens skapande. 1926-1927 deltog de i militära expeditioner för att underkuva krigsherrar i de norra delarna av landet, i samarbete med KMT, och expeditionen lyckades ena större delen av fastlandet under KMT. 1928 vände sig KMT mot kommunisterna och dödade och utvisade tusentals. Efter att Japan invaderade landerna tvingades KMT att samarbeta med kommunisterna mot Japan och krigsherrarna, men samarbetet gick dåligt eftersom parterna fokuserade på att motarbeta varandra. Under krigets gång blev KMT allt mindre populära på grund av sina brutala metoder för att kväva motstånd, medan kommunistpartiet blev allt mer populära, framförallt på landsbygden. Efter krigets slut träffades ledarna för att försöka samarbeta, men förhandlingarna ledde ingen vart och 1946 utbröt fullskaligt inbördeskrig.

Kommunistpartiet hade inte kontroll över någon av landets större städer men de kontrollerade landsbygden och hade mycket större folkligt stöd. Efter stora militära framgångar utropades Folkrepubliken Kina 1949 och KMT med anhängare samt militären och den ekonomiska eliten flydde till Taiwan, som hade återlämnats till Kina vid andra världskrigets slut. Inledningsvis skulle Taiwan bara vara en tillfällig bas, medan KMT förberrede sig för att återövra fastlandet, men när USA bestämde sig för att inte ingripa i inbördeskriget blev det uppenbart att det inte längre var möjligt. Koreakriget, där Folkrepubliken och USA stred på motsatta sidor, omöjliggjorde alla former av amerikanskt samarbete med kommunistregimen och ledde till att USA bestämde sig för att skydda regimen på Taiwan från en kommunistisk invasion och erkänna den som den enda legitima kinesiska regimen. 1979 bestämde man sig för att istället erkänna Folkrepubliken som den enda legitima regimen och avbröt officiella diplomatiska kontakter med regimen på Taiwan, men det militära stödet fortsatte.

Strax efter att KMT anlände till ön utlyste man undantagstillstånd och över 140 000 misstänkta kommunister eller regimmotståndare avrättades eller fängslades under den Vita Terrorn. Undantagstillståndet blev normaltillståndet och KMTs auktoritära styre fortsatte fram till 1980-talet, då demokratiska reformer påbörjades och andra partier tilläts.

När man anlände till ön tog man med sig hela den kinesiska valuta- och metallreserven samt stora delar av den intellektuella och ekonomiska eliten, detta i kombination med omfattande ekonomiskt stöd från USA gjorde att man snabbt kunde börja bygga upp sin ekonomi igen. Inledningsvis motsatte man sig import och hade endast statliga banker för att försöka bygga upp sin industri utan utländsk konkurrens, men på 60-talet började man liberalisera ekonomin. Låga löner, välutbildad befolkning, avsaknaden av regleringar och miljörelaterade lager samt bristen på facklig organisation attraherade företag från framförallt Japan och USA och ekonomin växte snabbt under hela andra halvan av 1900-talet. Idag är Republiken Kina en väl utvecklad ekonomi med en BNP per capita precis under Sveriges.

\section*{Globaliseringens Påverkan}
ROC har haft stora fördelar av globaliseringen, genom den accelererade globala ekonomin har ROC både fått en stark inkomst men också social och politisk relevans på den globala spelplanen. Den har blivit en stark aktör inom produktion. Innan globaliseringen så var ROC ansett ett utvecklingsland fattigdomenandelen var hög och arbetslösheten var omfattande. Efter globaliseringen har landet fått en blommande ekonomi med låg arbetslöshet och goda förutsättningar. Landet hade en arbetslöshet på 3.8\% år 2015.

ROC:s nutida industri härstammar huvudsakligen från den traditionella leksaksindustrin. Den utvecklades dock snabbt till att bli mer avancerad på den teknologiska fronten. Idag har ROC en export på varor värda cirka 285,000,000,000\$ per år (2015), vilket motsvara ca 61\% av Sveriges BNP eller 65\% av ROC:s BNP. Året innan var siffran 318,000,000,000\$. Minskningen är ett direkt resultat av den fallande efterfrågan på teknologi samt PRC:s avstannande.  

\section*{Läget idag}

\subsection*{Republiken Kina}
En väldigt viktig del av politiken i Republiken Kina (förkortat ROC), är den nationella identiteten. Huruvida invånarna främst är kineser eller taiwaneser är en mycket kontroversiell men viktig fråga. Innan demokratiseringen innehade kineser alla viktiga poster i samhället, de som hade bott på ön innan flykten från fastlandet hade väldigt lite inflytande och frågan om den nationella identiteten var tabubelagd, och har alltså viss etnisk laddning. KMTs ståndpunkt var att man var den enda legitima regimen i Kina och att hela Kinas territorium enligt 1949 års gränser tillhörde regimen. Inga andra åsikter var tillåtna och frågan var mycket känslig.

Först när andra politiska partier än KMT tilläts kunde andra åsikter bli hörda. Demokratiska framstegspartiet, DPP, bildades 1986 som ett alternativ till det styrande KMT. Inledningsvis undvek man frågan om Taiwans status, som fortfarande var mycket känslig, och fokuserade istället på demokrati, liberaliseringar och miljöfrågor. Allteftersom det politiska klimatet blev allt mer liberalt kunde man försiktigt lyfta frågan om självständighet och en taiwanesisk nationell identitet. 1991 tog partiet plats i parlamentet och 2000 vann man presidentval, det var första gången i republikens historia som presidenten inte tillhörde KMT. 2002 blev man största parti i parlamentet. Som en reaktion på detta bildades den Pan-blåa koalitionen, som bestod av KMT och andra partier som förespråkade en kinesisk identitet samt en återförening med fastlandet. På grund av att man inte hade en majoritet i parlamentet och på grund av presidentens begränsade makt kunde man inte åstadkomma så mycket. 2008 förlorade man både presidentvalet och parlamentsvalet, kandidaten från KMT blev president och den Pan-blåa koalitionen fick en majoritet i parlamentet.


\subsection*{Aktörer}

\subsubsection*{PRC}
PRC har en specifik myndighet dedikerad till att hantera situationen med ROC, kallat Kontoret för Taiwanesiska angelägenheter. Myndigheten har ansvaret för att verkställa alla beslut fattade av PRC:s regering, den har också ansvaret för de relevanta förberedelserna för förhandling med ROC.

\subsubsection*{Pan-gröna koalitionen}
Den Pan-gröna koalitionen står för att vara självständigt ifrån Fastlands-Kina.

\subsubsection*{Pan-blåa koalitionen}
Den Pan-blåa koalitionen arbetar, till skillnad från den Pan-gröna, för ett enat Kina.

\subsection*{Det som håller konflikten vid liv}
En inneboende nationalism hos både ROC och PRC leder till att konflikten får syre till sin låga. Med tanke på att båda länderna har ansett att HELA Kina skulle tillhöra dem så innebär det att de båda staterna anser det andra landet ockupera en del av deras land, det har haft en tendens att skapa konflikt genom historien. Det finns också många som skulle hävda att "vi och dem"-mentaliteten främjar konflikten.

\section*{Internationella effekter}
PRC gör internationella politiska relationer till ROC svåra. PRC har förbjudit ambassader med ROC, för andra stater, ifall de vill ha en ambassad i PRC. Det gör så att det är väldigt få länder som har officiella ambassader hos ROC. Det är bara 21 länder har officiella ambassader i Taipei. Många länder har istället ett så kallat Taipeiskt representationskontor. 

PRC är världens största ekonomi (räknat i PPP). De har extremt billig arbetskraft och en lång tradition av att arbeta med produktion. Det gör det väldigt svårt för resten av världen att inte handla med dem. Många storföretag har näst intill all sin produktion i landet. Det gör så att PRC kan ställa stora krav på resten av världen.

\subsection*{USA}
USA är ett exempel på ett land som har ett Taipeiskt representationskontor. USA har goda handelsrelationer till ROC. 2014 fördes lagstiftning som gjorde det lättare för USA att handla med ROC. USA erkänner dock bara ett Kina, men de erkänner inte att Taiwan tillhör Kina.

Det amerikanska storföretaget Walmart har börjat köpa in produkter ifrån Kinesiska aktörer, både från PRC och ROC. Vilket har gjort så att konkurrerande företag har behövt flytta sin produktion till andra billigare länder, som t.ex. ROC och PRC. Andra företag, som t.ex. Apple och Dell har också flyttat produktion till Kina.

USA har under lång tid varit mycket emot den kommunistiska ideologin. Under kalla kriget fick kommunismen en väldigt dålig karaktär i USA och de har därför satt sig på motsatt sida i många konflikter. På ett eller annat sätt kan det här ses som samma sak, även fast de har erkänt PRC som det enda Kina så har de mycket goda relationer med ROC (som Taiwan). Det kan bero på att PRC är en ''kommunistisk'' stat.

\subsection*{Europa}
Det enda landet i Europa som har erkänt ROC som det enda och sanna Kina är Vatikanstaten. Ungefär hälften av de resterande länderna har inofficiella ambassader (representationskontor) i ROC. Europa har, precis som USA, mycket produktion i Kina. Det gör att de Europeiska ekonomierna också är väldigt beroende av framförallt PRC, men även ROC. 

\subsection*{Syd- och Centralamerika}
Stora delar av Centralamerika har ambassader i ROC (dvs. får inte ha ambassad i PRC). Paraguay är dock det enda Sydamerikanska landet som erkänner ROC och dess anspråk.

\section*{Framtid}
Eftersom det största partiet i den Pan-gröna koalitionen har ensam absolut majoritet i ROC:s parlament från och med den 20 maj 2016 så ser den nära framtiden möjligt turbulent ut i den här konflikten. Senast som den Pan-gröna koalitionen styrde så byggde PRC upp en stor armada av 2000 missiler på andra sidan Taiwansundet.

Eftersom PRC och ROC har ingått i en överenskommelse, att de båda är en del av ett och samma Kina, så är det bara en fråga om tid innan det är enade på kartan också. Det finn realistiskt sätt bara två scenarion:

Det försa är att PRC och ROC ingår i en diplomatisk union där de två staterna enas. Idag ser det ut som om det skulle hända med hjälp av både käpp och morot. Företag i ROC har nyligen haft stor framgång i investering i PRC:s marknad. Moroten skulle innebär premier som gör det ännu mer gynnsamt med dessa investeringar, genom t.ex. billiga lån till företag baserade på Taiwan. Käppen för oss till det nästa scenariot, krig. Kombinationen av hotet att annektera Taiwan med våld och privilegierna de skulle få ROC anslöt sig till PRC är mycket möjligt tillräckligt för att få ROC att ansluta sig till PRC.

Det andra scenariot innebär ett nytt Kinesiskt inbördeskrig, eller snarare en fortsättning på det gamla. Under det senaste styret av den Pan-gröna koalitionen så såg det ut som ett mycket möjligt scenario då PRC snabbt rustade upp sin sida av Taiwansundet. USA har sedan1979 utrustat ROC:s armé med amerikansk stridsutrustning för att de ska kunna försvara sig. I nuläget är det inte ens nära tillräckligt för att försvara sig mot en invasion från folkets befrielsearmé (PRC:s armé). Det finns inte heller stor sannolikhet att USA skulle lägga sig i ett Kinesiskt inbördeskrig, med tanke på den ekonomiska skulden som USA har till båda parterna.

\section*{Källor}
\subsection*{\href{https://www.landguiden.se/Lander/Asien/Taiwan}{Landguiden}}
Urikespolitiska Institutet är ansvarig utgivare av Landguiden. Utrikespolitiska institutet är ett oberoende institut med syfte att forska och informera inom utrikespolitika frågor. Deras agenda är att främja intresse för utriketpolitiska frågor. Det kan innebära att de förvränger sanningen för att skapa mer intresse än vad sanningen erbjuder inom ämnet, men eftersom de finansieras av forskningsverksamhet och statligt så bör deras verksamhet bli granskad av dessa organisationer som strävar efter sanningen (med staten kanske det är lite tveksamt :P). Informationen på Ladguiden är hämtad från ett flertal källor som sammanvägs av UI för att få fram den mest troliga sanningen, därav borde inte någon felaktig information vara där på grund av ignorans eller okunnighet. 

Med tanke på att man behöver betala för att få fullständig tillgång till Landguiden och att den inte har något annat som lockar än information så skulle den tappa väldigt mycket pengar ifall det läckte att det var en partisk källa. Det är inte heller någon reklam vilket visar på att betaltjänsten är det ända sättet som sidan i sig får pengar.

Landguiden är regelbundet uppdaterad och ser väl underhållen ut.
\subsection*{\href{https://www.cia.gov/library/publications/the-world-factbook/geos/tw.html}{CIA World Factbook}}
CIA World Factbook är gjort av den amerikanska myndigheten CIA i syfte om att skapa en databas med fakta. Med tanke på att det är en amerikansk myndighets källa så bör den vara vinklad för att spegla på PRC negatict. Det har tagits i åtanke vid hämtande av info från den den här källan.

\subsection*{\href{http://isites.harvard.edu/fs/docs/icb.topic199080.files/Readings_for_October_23_/Chu.AS.04.pdf}{Taiwan’s national identity politics and the prospect of cross-strait relations, Yun-han Chu}}

% Kinas historia - jag
%
% läget idag - aktörer, ståndpunkter, vad som håller konflikten vid liv - jag
%
% hur relationerna påverkar omvärlden - hur omvärlden påverkar läget - du
%
% framtid -björn
%
%
% relationer idag - hjälp från USA
%
% hur relationerna påverkar omvärlden
%
% historia/inledning/bakgrund - introduktion
%
% politiska partier i roc - jag
%
% prc:s åsikter
%
% globaliseringens påverkan - du
%
%
% du utrikes + framtid
%
% jag inrikes + dåtid


\end{document}
