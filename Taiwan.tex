\documentclass[a4paper,10pt]{article}
\usepackage[top=1in,bottom=1in,left=1in,right=1in]{geometry}
\usepackage{fullpage}
\usepackage[english,swedish]{babel}
\usepackage[T1]{fontenc} 
\usepackage[utf8]{inputenc}
\usepackage{amsmath}
\usepackage{amssymb}
\usepackage{amsthm}
\usepackage{amsfonts}
\usepackage[pdftex]{graphicx}
\usepackage{fancyvrb}
\usepackage{array}
\usepackage{fancyhdr}
\usepackage{courier}
\usepackage{booktabs}
\usepackage{paralist}
\usepackage{xfrac}
\usepackage{siunitx} % Provides the \SI{}{} and \si{} command for typesetting SI units
\usepackage{graphicx} % Required for the inclusion of images
\usepackage{amsmath} % Required for some math elements 
\usepackage{tikz}
% \usetikzlibrary{arrows.meta}
% \usepackage{cite}
\usepackage{enumitem}
\usepackage{float}
\usepackage{wrapfig}
\usepackage{multicol}
\usepackage{caption}
\usepackage{svg}
% \usepackage{subcaption}
% \usepackage{natbib}
\usepackage[citestyle=verbose-ibid,bibstyle=numeric,backend=bibtex,sorting=nty]{biblatex}
\addbibresource{reference.bib}
\usepackage{csquotes}
\usepackage{pdfpages}
\usepackage{appendix}
\usepackage{url}
\usepackage{hyperref}
\usetikzlibrary{scopes}

\setlength\parindent{0pt}
%\urlstyle{same}

\bibliography{sample}

% Ger en titelsida för bilagor
\renewcommand*{\appendixpagename}{\newpage Bilagor} 
\renewcommand*{\appendixtocname}{Bilagor} 




% Gör titel
% \usepackage{titlesec}
% \titleformat{\section}{\normalfont\Large\bfseries}{\S \thesection}{1em}{}
\title{\Huge\bf{Taiwan}\\}
\author{\emph{Björn Thorén, Johannes Agestam}}
\date{\today}
\pagenumbering{gobble}


\begin{document}
%\pagenumbering{gobble}
%\null  % Empty line
%\nointerlineskip  % No skip for prev line
%\vfill
%\let\snewpage \newpage
%\let\newpage \relax
\maketitle
%\let \newpage \snewpage
%\vfill 
% \break % page break

%\newpage

\section*{Introduktion}

\section*{Historia}

\subsection*{slutet på kejsardömet till början av inbördeskriget}

\subsection*{inbördeskriget och världskriget fram till och med flykten till taiwan}

\subsection*{hjälp från usa och inbördeskrigets slut}

\subsection*{utveckling efter kriget (bara taiwan eller båda kina?)}


\section*{Läget idag}

% björn tar storkina, jag tar lillkina

\subsection*{Aktörer}

\subsubsection*{PRC}
PRC har en specifik myndighet dedikerad till att hantera situationen med ROC, kallat Kontoret för Taiwanesiska angelägenheter. Myndigheten har ansvaret för att verkställa alla beslut fattade av PRC:s regering, den har också ansvaret för de relevanta förberedelserna för förhandling med ROC.

\subsubsection*{Pan-gröna koalitionen}
Den Pan-gröna koalitionen står för att vara självständigt ifrån Fastlands-Kina.

\subsubsection*{Pan-blåa koalitionen}
Den Pan-blåa koalitionen arbetar, till skillnad från den Pan-gröna, för ett enat Kina.

\subsection*{Det som håller konflikten vid liv}
En inneboende nationalism hos både ROC och PRC leder till att konflikten får syre till sin låga. Med tanke på att båda länderna har ansett att HELA Kina skulle tillhöra dem så innebär det att de båda staterna anser det andra landet ockupera en del av deras land, det har haft en tendens att skapa konflikt genom historien. Det finns också många som skulle hävda att "vi och dem"-mentaliteten främjar konflikten.

\section*{Internationella effekter}
PRC gör internationella relationer till ROC svåra. PRC har förbjudit ambassader med ROC, för andra stater, ifall de vill ha en ambassad i PRC. Det gör så att det är väldigt få länder som har officiella ambassader hos ROC. Det är bara 21 länder har officiella ambassader i Taipei. Många länder har istället ett så kallat Taipeiskt representationskontor.

\subsection*{USA}
USA är ett exempel på ett land som har ett Taipeiskt representationskontor. USA har goda handelsrelationer till ROC. 2014 fördes lagstiftning som gjorde det lättare för USA att handla med ROC. USA erkänner dock bara ett Kina, men de erkänner inte att Taiwan tillhör Kina. 

USA har under lång tid varit mycket emot den kommunistiska ideologin. Under kalla kriget fick kommunismen en väldigt dålig karaktär i USA och de har därför satt sig på motsatt sida i många konflikter. På ett eller annat sätt kan det här ses som samma sak, även fast de har erkänt PRC som det enda Kina så har de mycket goda relationer med ROC (som Taiwan). Det kan bero på att PRC är en ''kommunistisk'' stat.

\subsection*{Europa}
Det enda landet i Europa som har erkänt ROC som det enda och sanna Kina är Vatikanstaten. Ungefär hälften av de resterande länderna har inofficiella ambassader (representationskontor) i ROC.

\subsection*{Syd- och Centralamerika}
Stora delar av Centralamerika har ambassader i ROC (dvs. får inte ha ambassad i PRC). Paraguay är dock det enda Sydamerikanska landet som erkänner ROC och dess anspråk.

\section*{Framtid}
Eftersom det största partiet i den Pan-gröna koalitionen har ensam absolut majoritet i Taiwans parlament från och med den 20 maj 2016 så ser den nära framtiden möjligt turbulent ut i den här konflikten. Eftersom senast som den Pan-gröna koalitionen styrde så byggde PRC upp en stor armada av missiler på andra sidan Taiwansundet. Det kommer med absolut största sannolikhet innebära att det blir någon form av upprustning i området. 

\section*{Källor}
\subsection*{\href{https://www.landguiden.se/Lander/Asien/Taiwan}{Landguiden}}
Landguiden är  

\href{https://www.cia.gov/library/publications/the-world-factbook/geos/tw.html}{CIA Factbook}\\
\href{http://isites.harvard.edu/fs/docs/icb.topic199080.files/Readings_for_October_23_/Chu.AS.04.pdf}{Taiwan’s national identity politics and the prospect of cross-strait relations, Yun-han Chu}\\

% Kinas historia - jag
%
% läget idag - aktörer, ståndpunkter, vad som håller konflikten vid liv - jag
%
% hur relationerna påverkar omvärlden - hur omvärlden påverkar läget - du
%
% framtid -björn
%
%
% relationer idag - hjälp från USA
%
% hur relationerna påverkar omvärlden
%
% historia/inledning/bakgrund - introduktion
%
% politiska partier i roc - jag
%
% prc:s åsikter
%
% globaliseringens påverkan - du
%
%
% du utrikes + framtid
%
% jag inrikes + dåtid


\end{document}
