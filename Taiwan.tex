\documentclass[a4paper,10pt]{article}
\usepackage[top=1in,bottom=1in,left=1in,right=1in]{geometry}
\usepackage{fullpage}
\usepackage[english,swedish]{babel}
\usepackage[T1]{fontenc} 
\usepackage[utf8]{inputenc}
\usepackage{amsmath}
\usepackage{amssymb}
\usepackage{amsthm}
\usepackage{amsfonts}
\usepackage[pdftex]{graphicx}
\usepackage{fancyvrb}
\usepackage{array}
\usepackage{fancyhdr}
\usepackage{courier}
\usepackage{booktabs}
\usepackage{paralist}
\usepackage{xfrac}
\usepackage{siunitx} % Provides the \SI{}{} and \si{} command for typesetting SI units
\usepackage{graphicx} % Required for the inclusion of images
\usepackage{amsmath} % Required for some math elements 
\usepackage{tikz}
\usetikzlibrary{arrows.meta}
% \usepackage{cite}
\usepackage{enumitem}
\usepackage{float}
\usepackage{wrapfig}
\usepackage{multicol}
\usepackage{caption}
\usepackage{svg}
% \usepackage{subcaption}
% \usepackage{natbib}
\usepackage[citestyle=verbose-ibid,bibstyle=numeric,backend=bibtex,sorting=nty]{biblatex}
\addbibresource{reference.bib}
\usepackage{csquotes}
\usepackage{pdfpages}
\usepackage{appendix}
\usepackage{url}
\usetikzlibrary{scopes}

\setlength\parindent{0pt}
\urlstyle{same}

\bibliography{sample}

% Ger en titelsida för bilagor
\renewcommand*{\appendixpagename}{\newpage Bilagor} 
\renewcommand*{\appendixtocname}{Bilagor} 

% Gör titel
\usepackage{titlesec}
% \titleformat{\section}{\normalfont\Large\bfseries}{\S \thesection}{1em}{}
\title{\Huge\bf{Taiwan}\\}
\author{\emph{Björn Thorén, Johannes Agestam}}
\date{2016-04-12}
\pagenumbering{gobble}


\begin{document}
\pagenumbering{gobble}
\null  % Empty line
\nointerlineskip  % No skip for prev line
\vfill
\let\snewpage \newpage
\let\newpage \relax
\maketitle
\let \newpage \snewpage
\vfill 
\break % page break

\newpage


\section{Historia}

\subsection{slutet på kejsardömet till början av inbördeskriget}

\subsection{inbördeskriget och världskriget fram till och med flykten till taiwan}

\subsection{hjälp från usa och inbördeskrigets slut}

\subsection{utveckling efter kriget (bara taiwan eller båda kina?)}


\section{Läget idag}

\subsection{aktörer}

\subsection{ståndpunkter}

\subsection{vad som håller konflikten vid liv}




kinas historia - jag

läget idag - aktörer, ståndpunkter, vad som håller konflikten vid liv - jag

hur relationerna påverkar omvärlden - hur omvärlden påverkar läget - du

framtid -björn


relationer idag - hjälp från usa

hur relationerna påverkar omvärlden

historia/inledning/bakgrund - introduktion

politiska partier i roc - jag

prc:s åsikter

globaliseringens påverkan - du


du utrikes + framtid

jag inrikes + dåtid


\end{document}
